% \iffalse
\let\negmedspace\undefined
\let\negthickspace\undefined
\documentclass[journal,12pt,twocolumn]{IEEEtran}
\usepackage{cite}
\usepackage{amsmath,amssymb,amsfonts,amsthm}
\usepackage{algorithmic}
\usepackage{graphicx}
\usepackage{textcomp}
\usepackage{xcolor}
\usepackage{txfonts}
\usepackage{listings}
\usepackage{enumitem}
\usepackage{mathtools}
\usepackage{gensymb}
\usepackage{comment}
\usepackage[breaklinks=true]{hyperref}
\usepackage{tkz-euclide} 
\usepackage{listings}
\usepackage{gvv}
%\def\inputGnumericTable{} 
\usepackage[latin1]{inputenc}
\usepackage{color}                                            
\usepackage{array}                                            
\usepackage{longtable}                                       
\usepackage{calc}                                             
\usepackage{multirow}                                         
\usepackage{hhline}                                           
\usepackage{ifthen}                                           
\usepackage{lscape}

\newtheorem{theorem}{Theorem}[section]
\newtheorem{problem}{Problem}
\newtheorem{proposition}{Proposition}[section]
\newtheorem{lemma}{Lemma}[section]
\newtheorem{corollary}[theorem]{Corollary}
\newtheorem{example}{Example}[section]
\newtheorem{definition}[problem]{Definition}
\newcommand{\BEQA}{\begin{eqnarray}}
\newcommand{\EEQA}{\end{eqnarray}}
\newcommand{\define}{\stackrel{\triangle}{=}}
\theoremstyle{remark}
\newtheorem{rem}{Remark}
\begin{document}

\bibliographystyle{IEEEtran}
\vspace{3cm}

\title{Chapter-2 Complex Number}
\author{AI24BTECH11032 Shreyansh Sonkar
}
\maketitle
\newpage
\bigskip

\renewcommand{\thefigure}{\theenumi}
\renewcommand{\thetable}{\theenumi}
\begin{enumerate}




\item[7.] If $1,a_1,a_2,a_3....a_{n-1}$
 are the n roots of unity, then show that $\brak{1-a_1}\brak{1-a_2}\brak{1-a_3}....\brak{1-a_{n-1}}$=n  
 
  \hfill (1984- 2 Marks).\\



\item[8.]  Show that the area of the 
  triangle on the Argand 
  
  diagram formed by the complex 
  numbers
  
  $z,iz \And z+iz$ 
  is $\frac{1}{2}|z|^2$.
  
  \hfill (1986-2 Marks).\\



\item[9.] Let $Z_1=10+6\iota$ and $Z_2=4+6\iota$. if $Z$ is any 

complex number such that the argument of ${\frac{\brak{Z-Z_1}}{\brak{Z-Z_2
}}}$

is  ${\frac{\pi}{4}}$ then prove that $Z-7-9\iota =3\sqrt{2}$
  
  \hfill(1990-4 Marks).\\


\item[10.] if {$iz^{3} -z^2-z+\iota=0$} then show that $|z| =1$

\hfill(1995-5 Marks).\\


\item[11.] If $|Z|\le1,|W|\le1$,show that

$\brak{Z-W}^2\le\brak{|Z|-|W|}^2+\brak{\arg Z- \arg W}^2$ 

\hfill(1995-5 Marks). \\


\item[12.] Find all non-zero complex numbers Z 

satisfying $\bar Z=\iota Z^2 $ 

\hfill(1996-2 Marks).\\

\item[13.] Let $z_1$ and $z_2$ be roots of the equation 

$z^2+pz+q=0$, where the coefficients p and q

may be complex numbers. Let A and B represent 

$z_1$ and $z_2$ in the complex plane . if$ \angle AOB = \alpha \not=0$ 

and $OA=OB$,where O is the origin,prove that 

$p^2=4q\cos^2\brak{\frac{\alpha}{2}}$. 

\hfill(1997-5 Marks)\\

\item[14.] For complex number z and w,prove 

that $|z|^2w-|w|^2z=z- w$  if and only 

if $ z = w or \bar w = 1$.
   
   \hfill(1999-10 Marks).\\


\item[15.] Let a complex number $\alpha$ ,$\alpha \not=1$, be a root of 

the equation $z^{p+q}-z^p-z^q+1=0$, where p,q are 

the distinct primes. Show that either 

$1+\alpha+\alpha^2+...+\alpha^{p-1}=0$ or $1+\alpha+\alpha^2+...+\alpha^{q- 1}=0$,

but not both together. 

\hfill(2002-5 Marks)\\

\item[16.] If $z_1$and$z_2$ are two  complex number such that

     $|z_1<1<||z_2|$ then prove that $|\frac{1-z_1\bar z_2}{z_1-z_2}|<1.$ 
    
    \hfill(2003-2 Marks)\\

\item[17.] Prove that there exists no  complex number z 

      such that $|z|<\frac{1}{3}$ and  $\sum_{r=1}^{n} a_r z^r =1$ where $|a_r|<2$.
   
  
   \hfill (2003-2 Marks)\\
      
      
\item[18.] Find the centre and radius  of circle given by

    $|\frac{z-\alpha}
    {z-\beta}|=k$,$k\not=1$ 
    where,$z=x+\iota,\alpha=\alpha_1+\iota \alpha_2  $, 
    
    $\beta=\beta_1+\iota \beta_2$. 
    
    \hfill(2004-2 Marks) \\
      
\item[19.] If one the vertices of the square 

circumscribing the circle $|z-1|= \sqrt{2}$ is $2+\sqrt{3} \iota$.

find the other vertices of the square .

\hfill (2005-4 Marks) \\

\end{enumerate}
\end{document}
