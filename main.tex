% \iffalse
\let\negmedspace\undefined
\let\negthickspace\undefined
\documentclass[journal,12pt,twocolumn]{IEEEtran}
\usepackage{cite}
\usepackage{amsmath,amssymb,amsfonts,amsthm}
\usepackage{algorithmic}
\usepackage{graphicx}
\usepackage{textcomp}
\usepackage{xcolor}
\usepackage{txfonts}
\usepackage{listings}
\usepackage{enumitem}
\usepackage{mathtools}
\usepackage{gensymb}
\usepackage{comment}
\usepackage[breaklinks=true]{hyperref}
\usepackage{tkz-euclide} 
\usepackage{listings}
\usepackage{gvv}                                        
\def\inputGnumericTable{}                                 
\usepackage[latin1]{inputenc}                                
\usepackage{color}                                            
\usepackage{array}                                            
\usepackage{longtable}                                       
\usepackage{calc}                                             
\usepackage{multirow}                                         
\usepackage{hhline}                                           
\usepackage{ifthen}                                           
\usepackage{lscape}

\newtheorem{theorem}{Theorem}[section]
\newtheorem{problem}{Problem}
\newtheorem{proposition}{Proposition}[section]
\newtheorem{lemma}{Lemma}[section]
\newtheorem{corollary}[theorem]{Corollary}
\newtheorem{example}{Example}[section]
\newtheorem{definition}[problem]{Definition}
\newcommand{\BEQA}{\begin{eqnarray}}
\newcommand{\EEQA}{\end{eqnarray}}
\newcommand{\define}{\stackrel{\triangle}{=}}
\theoremstyle{remark}
\newtheorem{rem}{Remark}
\begin{document}

\bibliographystyle{IEEEtran}
\vspace{3cm}

\title{Chapter-2 Complex Number}
\author{AI24BTECH11032-Shreyansh Sonkar
}
\maketitle
\newpage
\bigskip

\renewcommand{\thefigure}{\theenumi}
\renewcommand{\thetable}{\theenumi}
$$PASSAGE-1$$

    


 Let $\vec{A}$,$\vec{B}$,$\vec{C}$ be three sets of complex number as\\ 
  defined below\\
$\vec{A}$=$\{z:Im z\geq 1\}$\\
$\vec{B}$=$\{z:|z-2-\iota|=3\}$\\
$\vec{C}$=$\{z:Re\brak{\brak{1-\iota}z}=\sqrt{2}\}$\\
\begin{enumerate}
    

\item The number of element in the set$\vec{A} \cap\vec{B} \cap\vec{C} $

is \hfill (2008)\\

$\brak{a} 0 \ \brak{b} 1 \  \brak{c} 2 \ \brak{d}\infty \ $ \\

\item Let z be any point in$ A\cap B \cap C$

 Then,$|z+1-\iota|^2+|z-5-\iota|^2$ lies between
 
 \hfill  (2008) \\
             
 
  \brak{a} 25 and 29 \  \brak{b} 30 and 34\\
 
 \brak{c} 35 and 39 \  \brak{d}40 and 44 \\

\item  Let z be any point A B C and let w be any

point satisfying $|w-2-\iota|<3$.Then, $|z|-|w|+3$ lies between
   
   \hfill (2008)

 \brak{a} -6 and 3\   \brak{b} -3 and 6 \\

 \brak{c} -6 and 6\   \brak{d} -3 and 9\\
 \end{enumerate}

 
\end{document}