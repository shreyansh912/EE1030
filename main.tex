\let\negmedspace\undefined
\let\negthickspace\undefined
\documentclass[journal]{IEEEtran}
\usepackage[a5paper, margin=10mm, onecolumn]{geometry}
%\usepackage{lmodern} % Ensure lmodern is loaded for pdflatex
\usepackage{tfrupee} % Include tfrupee package

\setlength{\headheight}{1cm} % Set the height of the header box
\setlength{\headsep}{0mm}     % Set the distance between the header box and the top of the text

\usepackage{gvv-book}
\usepackage{gvv}
\usepackage{cite}
\usepackage{amsmath,amssymb,amsfonts,amsthm}
\usepackage{algorithmic}
\usepackage{graphicx}
\usepackage{textcomp}
\usepackage{xcolor}
\usepackage{txfonts}
\usepackage{listings}
\usepackage{enumitem}
\usepackage{mathtools}
\usepackage{gensymb}
\usepackage{comment}
\usepackage[breaklinks=true]{hyperref}
\usepackage{tkz-euclide} 
\usepackage{listings}
% \usepackage{gvv}                                        
\def\inputGnumericTable{}                                 
\usepackage[latin1]{inputenc}                                
\usepackage{color}                                            
\usepackage{array}                                            
\usepackage{longtable}                                       
\usepackage{calc}                                             
\usepackage{multirow}                                         
\usepackage{hhline}                                           
\usepackage{ifthen}                                           
\usepackage{lscape}
\begin{document}

\bibliographystyle{IEEEtran}
\vspace{3cm}

\title{10.5.3.9}
\author{EE23BTECH11063 - Vemula Siddhartha
}



\renewcommand{\thefigure}{\theenumi}
\renewcommand{\thetable}{\theenumi}
\setlength{\intextsep}{10pt} % Space between text and floats


\numberwithin{equation}{enumi}
\numberwithin{figure}{enumi}
\renewcommand{\thetable}{\theenumi}

\title{Chapter-2 Complex Number}
\author{AI24BTECH11032 Shreyansh Sonkar
}
\maketitle
\renewcommand{\thefigure}{\theenumi}
\renewcommand{\thetable}{\theenumi}
\begin{enumerate}[start=7] % This line has been modified to start from number 7\\
\item If $1$,$a_1$,$a_2$,$a_3$....$a_{n-1}$ are the $n$  roots of unity,then show that $\brak{1-a_1}\brak{1-a_2}\brak{1-a_3}....\brak{1-a_{n-1}}$ = $n$ \hfill (1984- 2 Marks).\\
\item Show that the area of the triangle on the Argand diagram formed by the complex numbers\\
$z,iz \And z+iz$ is $\frac{1}{2}\abs{z}^2$.\hfill (1986-2 Marks).\\
\item Let $Z_1$=$10+6i$ and $Z_2$=$4+6i$.if $Z$ is any complex number such that the argument of ${\frac{\brak{Z-Z_1}}{\brak{Z-Z_2}}}$ is ${\frac{\pi}{4}}$ then prove that $Z-7-9i$ = $3\sqrt{2}$.\hfill(1990-4 Marks).\\
\item if {$iz^{3}$+$z^2$-$z$+$i=0$} then show that $\abs{z}$ =$1$.\hfill(1995-5 Marks).\\
\item If $\abs{Z}\le1$,$\abs{W}\le1$,show that $\brak{z-w}^2\le\brak{\abs{Z}\abs{W}}^2+\brak{\arg Z-\arg W}^2$ .\hfill(1995-5 Marks). \\
\item Find all non-zero complex numbers $Z$ satisfying $\bar Z$=$ iZ^2 $ .\hfill(1996-2 Marks).\\
\item Let $z_1$ and $z_2$ be roots of the equation $z^2$+$pz$+$q$=$0$, where the coefficients $p$ and $q$ may be complex numbers.Let $\Vec{A}$ and $\Vec{B}$ represent $z_1$ and $z_2$ in the complex plane . if$\angle ABC$ = $\alpha \not=0$ and $OA$=$OB$,where$O$ is the origin,prove that $p^2$=$4q\cos^2\brak{\frac{\alpha}{2}}$. \hfill(1997-5 Marks)\\
\item For complex number $z$ and $w$,prove that $\abs{z}^2w$-$\abs{w}^2z$=$z$- $w$ if and only if $z$ =$w$ or $\bar{we} $ = $1$.\hfill(1999-10 Marks).\\
\item Let a complex number $\alpha$ ,$\alpha \not=1$, be a root of the equation $z^{p+q}$-$z^p$-$z^q$+$1$=$0$, where $\Vec{p}$,$\Vec{q}$ are the distinct primes. Show that either $1+\alpha+\alpha^2+...+\alpha^{p-1}=0$ or $1+\alpha+\alpha^2+...+\alpha^{q- 1}=0$,but not both together. \hfill(2002-5 Marks)\\
\item If $z_1$ and $z_2$ are two  complex number such that $\abs{z_1}<1<\abs{z_2}$ then prove that  $\abs{\frac{1-z_1 \bar z_2}{z_1-z_2}}<1$.\hfill(2003-2 Marks)\\
\item Prove that there exists no complex number $z$ such that $\abs{z}<\frac{1}{3}$ and $\sum_{r=1}^{n}$ $a_r$ $z^r$ =$1$ where $\abs{a_r}<2$.\hfill (2003-2 Marks)\\
\item Find the centre and radius  of circle given by $\abs{\frac{z-\alpha} b}$=$k$,$k\not=1$ where,$z=x+i$,$\alpha$=$\alpha_1$+$i\alpha_2$,
$b$=$b_1$+$ib_2$.\hfill(2004-2 Marks) \\  \item If one the vertices of the square circumscribing the circle $\abs{z-1} = \sqrt{2}$ is $2 +\sqrt{3}i$find the other vertices of the square .\hfill (2005-4 Marks) \\
$$PASSAGE-1$$
Let $\vec{A}$,$\vec{B}$,$\vec{C}$ be three sets of complex number as\\ defined below\\$\vec{A} = \{ $z$ :\mid \operatorname{Im}z\geq1\}$\\$\vec{B}$=$\{$z$:|z-2-i|=3\}$\\$\vec{C}$=\{$z$ : $\operatorname{Re} \left((1-i)z\right) = \sqrt{2} \}$\\
\begin{enumerate}     
\item The number of element in the set$\vec{A} \cap\vec{B} \cap\vec{C} $ is.\hfill (2008)\\
   (a) 0 \hfill
   (b) 1 \hfill
   (c) 2 \hfill
   (d) $\infty$\\
\item Let $\vec{z}$ be any point in$ \vec{A}\cap\vec{B}\cap\vec{C} $ Then,$|z+1-i|^2+|z-5-i|^2$ lies between. \hfill  (2008) 
\begin{multicols}{2}
    \begin{enumerate}
        \item 25 and 29
        \item 30 and 34
        \item 35 and 39
        \item 40 and 44
    \end{enumerate}
\end{multicols}
\item  Let $z$ be any point $\vec{A}$ $\vec{B}$ $\vec{C}$ and let $w$ be any\\point satisfying $|w-2-i|<3$.Then, $|z|-|w|+3$ lies between.\hfill (2008) 
 \begin{multicols}{2}
  \begin{enumerate}
    \item -6 and 3
    \item -3 and 6 
    \item -6 and 6
    \item -3 and 9
  \end{enumerate}
 \end{multicols}
\end{enumerate}
\end{enumerate}
\end{document}