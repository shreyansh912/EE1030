\let\negmedspace\undefined
\let\negthickspace\undefined
\documentclass[journal]{IEEEtran}
\usepackage[a5paper, margin=10mm, onecolumn]{geometry}
%\usepackage{lmodern} % Ensure lmodern is loaded for pdflatex
\usepackage{tfrupee} % Include tfrupee package

\setlength{\headheight}{1cm} % Set the height of the header box
\setlength{\headsep}{0mm}     % Set the distance between the header box and the top of the text

\usepackage{gvv-book}
\usepackage{gvv}
\usepackage{cite}
\usepackage{amsmath,amssymb,amsfonts,amsthm}
\usepackage{algorithmic}
\usepackage{graphicx}
\usepackage{textcomp}
\usepackage{xcolor}
\usepackage{txfonts}
\usepackage{listings}
\usepackage{enumitem}
\usepackage{mathtools}
\usepackage{gensymb}
\usepackage{comment}
\usepackage[breaklinks=true]{hyperref}
\usepackage{tkz-euclide} 
\usepackage{listings}

% \usepackage{gvv}                                        
\def\inputGnumericTable{}                                 
\usepackage[latin1]{inputenc}                                
\usepackage{color}                                            
\usepackage{array}                                            
\usepackage{longtable}                                       
\usepackage{calc}                                             
\usepackage{multirow}                                         
\usepackage{hhline}                                           
\usepackage{ifthen}                                           
\usepackage{lscape}
\begin{document}

\bibliographystyle{IEEEtran}
\vspace{3cm}


\renewcommand{\thefigure}{\theenumi}
\renewcommand{\thetable}{\theenumi}
\setlength{\intextsep}{10pt} % Space between text and floats


\numberwithin{equation}{enumi}
\numberwithin{figure}{enumi}
\renewcommand{\thetable}{\theenumi}

\title{JEE Mains PYQ 27/07/2021 Shift-1}
\author{AI24BTECH11032 Shreyansh Sonkar
}
\maketitle
\renewcommand{\thefigure}{\theenumi}
\renewcommand{\thetable}{\theenumi}
\begin{enumerate}[start=16] % This line has been modified to start from number 16\\

\item Let $f : \mathbf{R} \to \mathbf{R}$ be a function such that $f\brak{2}=4$ and $f'\brak{2}=1.$ Then the value of $\lim_{x \to 2} \frac{x^2 f\brak{2} - 4 f\brak{x}}{x - 2}$ is equal to :
\begin{multicols}{4}
    \begin{enumerate}
        \item $4$
        \item $8$
        \item $16$
        \item $12$
    \end{enumerate}
\end{multicols}
\bigskip
\item Let P and Q be two distinct points on a circle which has centre at $C\brak{2,3}$ and which passes through origin O .If OC is perpendicular to both the line segment CP and CQ then the set $\cbrak{P,Q}$ is equal to

    \begin{enumerate}
        \item $\cbrak{\brak{4,0},{\brak{0,6}}}$
        \item $\cbrak{\brak{2+2\sqrt{2},3-\sqrt{5}},{\brak{2-2\sqrt{2},3+\sqrt{5}}}}$
        \item $\cbrak{\brak{2+2\sqrt{2},3+\sqrt{5}},{\brak{2-2\sqrt{2},3-\sqrt{5}}}}$
        \item $\cbrak{\brak{-1,5},{\brak{5,1}}}$
    \end{enumerate}
\bigskip
\item let $\alpha$ , $\beta$ be two roots of the equation $x^{2} +\brak{20}^{\frac{1}{4}}x+\brak{5}^{\frac{1}{2}} = 0 .$ Then $\alpha^{8}+\beta^{8}$ is equal to
\begin{multicols}{4}
    \begin{enumerate}
        \item $10$
        \item $100$
        \item $50$
        \item $160$
    \end{enumerate}
\end{multicols}
\bigskip
\item The probability that a randomly selected $2-\text{digit}$ number belongs to the set $\cbrak{n \in \mathbf{N} : \brak{2^{n} - 2} \text{is a multiple of } 3}$ is equal to 
\begin{multicols}{4}
    \begin{enumerate}
        \item $\frac{1}{6}$
        \item $\frac{2}{3}$
        \item $\frac{1}{2}$
        \item $\frac{1}{3}$
    \end{enumerate}
\end{multicols}
\bigskip
\item Let 
\begin{align*}
    A = \cbrak{\brak{x,y} \in \mathbf{R}\times\mathbf{R}\mid 2x^{2}+2y^{2}-2x-2y=1},\\
    B = \cbrak{\brak{x,y} \in \mathbf{R}\times\mathbf{R}\mid 4x^{2}+4y^{2}-16y+7=0} and\\
    C = \cbrak{\brak{x,y} \in \mathbf{R}\times\mathbf{R}\mid x^{2}+y^{2}-4x-2y+5\leq r^{2}},
\end{align*}
Then the minimum value of $\abs{r}$ such that $A \cup B \subseteq C$ is equal  to 
\begin{multicols}{2}
    \begin{enumerate}
        \item $\frac{3+\sqrt{10}}{2}$
        \item $\frac{2+\sqrt{10}}{2}$
        \item $\frac{3+2\sqrt{10}}{2}$
        \item $1+\sqrt{5}$
    \end{enumerate}
\end{multicols}
\bigskip
\item For real number $\alpha$ and $\beta$ , consider the following system of linear equation:
$x+y-z=2,x+2y+\alpha z=1,2x-y+z=\beta.$ If the system has infinite solution ,then $\alpha+\beta$ is equal to 
\bigskip
\item Let $  \overrightarrow{a} = \hat{i} + \hat{j} + \hat{k}
$ and $\overrightarrow{b}$ and $\vec{c} =  \hat{j} - \hat{k}$
 be three vectors such that $\overrightarrow{a} \times \overrightarrow{b}=\overrightarrow{c}$ and $\overrightarrow{a}\cdot \overrightarrow{b} =1.$ If the length of projection vector of the vector $\overrightarrow{b}$ on the vector $\overrightarrow{a} \times \overrightarrow{c}$
 is l, then the value of $3l^{2}$ is equal to \bigskip
 \item if $\log_3 2, \log_3\brak{2^{x}-5},\log_3\brak{2^{x}-\frac{7}{2}}$ are in an arithmetic progression, then the value of x is equal to 
\bigskip
\item Let the domain of the function $f\brak{x}=\log_4\brak{\log_5\brak{log_3\brak{18x-x^{2}-77}}}$ be $\brak{a,b}.$ Then the value of the integral $\int_{a}^{b} \frac{\sin^{3} x}{\brak{\sin^{3} x + \sin^{3}\brak{a + b - x}}}\, dx$ is equal to 
 \bigskip
 \item Let \[
f(x) = \begin{bmatrix}
\sin^{2} x & -2 + \cos^{2} x & \cos 2x \\
2 + \sin^{2} x & \cos^{2}x & \cos 2x \\
\sin^{2} x & \cos^{2} x & 1 + \cos 2x
\end{bmatrix}, \quad x \in \sbrak{0, \pi}
\]
Then the maximum value of $f\brak{x} $ is equal to \bigskip
\item Let $F :\sbrak{3, 5} \to \mathbf{R}$ be a twice differentiable function on $\brak{3,5}$ such that \[F\brak{x}=F(x) = e^{-x} \int_{3}^{x} (3t^{2} + 2t + 4F^{'}\brak{t}) \, dt\] If $ F^{'}\brak{4}=\frac{\alpha e^{\beta}-224}{\brak{e^{\beta}-4}^{2}},$ then $\alpha+\beta$ is equal to.
\bigskip
\item Let a plane P pass through the point $\brak{3,7,-7}$ and contain the line, $\frac{x-2}{-3}=\frac{y-3}{2}=\frac{z+2}{1}.$ If distance of the plane P from the origin is d,then $d^{2}$ is equal to .\bigskip
\item  Let $S = \cbrak{1,3,4,5,6,7}.$ Then the number of possible function $f:\mathbb{S}\to \mathbb{S}$ such that $f\brak{m.n}=f\brak{m}.f\brak{n}$ for every $m,n\in S$ and $m.n\in S$ is equal to .\bigskip
\item If $y=y\brak{x},y \in [0,\frac{\pi}{2})$ is the solution of the differential equation \[
\sec(y) \frac{d}{dx}(y) - \sin(x + y) - \sin(x - y) = 0\] then $5y^{'}\brak{\frac{\pi}{2}}$ is equal to.
\bigskip
\item Let $f:\sbrak{0,3}\in \mathbf{R}$ be defined by
\begin{align*}
f\brak{x}=min\cbrak{x-\sbrak{x},1+\sbrak{x}-x}
\end{align*}
where $\sbrak{x}$ is the greatest integer less than or equal to x.Let P denote the set containing all $x\in\sbrak{0,3}$ where f is discontinuous , and Q denote the set containing all $x\in\brak{0,3}$ where f is not differentiable .Then the sum of number of element in P and Q is equal to.
 



   

\end{enumerate}
\end{document}


