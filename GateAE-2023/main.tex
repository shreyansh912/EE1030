\let\negmedspace\undefined
\let\negthickspace\undefined
\documentclass[journal]{IEEEtran}
\usepackage[a5paper, margin=10mm, onecolumn]{geometry}
%\usepackage{lmodern} % Ensure lmodern is loaded for pdflatex
\usepackage{tfrupee} % Include tfrupee package

\setlength{\headheight}{1cm} % Set the height of the header box
\setlength{\headsep}{0mm}     % Set the distance between the header box and the top of the text

\usepackage{gvv-book}
\usepackage{gvv}
\usepackage{cite}
\usepackage{amsmath,amssymb,amsfonts,amsthm}
\usepackage{algorithmic}
\usepackage{graphicx}
\usepackage{textcomp}
\usepackage{xcolor}
\usepackage{txfonts}
\usepackage{listings}
\usepackage{enumitem}
\usepackage{mathtools}
\usepackage{gensymb}
\usepackage{comment}
\usepackage[breaklinks=true]{hyperref}
\usepackage{tkz-euclide} 
\usepackage{listings}
% \usepackage{gvv}                                        
\def\inputGnumericTable{}                                 
\usepackage[latin1]{inputenc}                                
\usepackage{color}                                            
\usepackage{array}                                            
\usepackage{longtable}                                       
\usepackage{calc}                                             
\usepackage{multirow}                                         
\usepackage{hhline}                                           
\usepackage{ifthen}                                           
\usepackage{bookmark}
\usepackage{lscape}

\begin{document}

\bibliographystyle{IEEEtran}
\vspace{3cm}

\title{Gate AE-2023}
\author{AI24BTECH11032 Shreyansh Sonkar}
% \maketitle
% \newpage
% \bigskip
{\let\newpage\relax\maketitle}

\renewcommand{\thefigure}{\theenumi}
\renewcommand{\thetable}{\theenumi}
\setlength{\intextsep}{10pt} % Space between text and floats


\numberwithin{equation}{enumi}
\numberwithin{figure}{enumi}
\renewcommand{\thetable}{\theenumi}

\begin{enumerate} [start=27]
\item Which of the following statement $\brak{\text{ s }}$ is/are true about the ribs of an airplane wing with semi-monocoque construction?
\begin{enumerate}
    \item For a rectangular planform wing, the dimensions of the ribs DO NOT depend on their spanwise position in the wing.
    \item Ribs increase the column buckling stress of longitudinal stiffeners connected to them.
    \item Ribs increase plate buckling stress of the skin panels.
    \item Ribs help in maintaining aerodynamic shape of the wing.
\end{enumerate}
\bigskip
\item From the options given, select all that are true for turbofan engines with
afterburners. 
\begin{enumerate}
    \item Turning afterburner ON increases specific fuel consumption.
    \item Turbofan engines with afterburners have variable area nozzles.
    \item Turning afterburner ON decreases specific fuel consumption.
    \item Turning afterburner ON increases stagnation pressure across the engine.
\end{enumerate}
\bigskip
\item Which of the following statement $\brak{\text{ s }}$ is/are true with respect to eigenvalues and eigenvectors of a matrix?
\begin{enumerate}
    \item The sum of the eigenvalues of a matrix equals the sum of the elements of the principal diagonal.
    \item If $\lambda$ is an eigenvalue of a matrix A $\frac{1}{\lambda}$ is always an eigenvalue of its transpose $\brak{A^{T}}.$
    \item If $\lambda$ is an eigenvalue of an orthogonal matrix A $\frac{1}{\lambda}$ is always an eigenvalue of  A.
    \item If a matrix has n distinct eigenvalues, it also has n independent eigenvectors. 
\end{enumerate}
\bigskip
\item For studying wing vibrations, a wing of mass M and finite dimensions has been
idealized by assuming it to be supported using a linear spring of equivalent
stiffness k and a torsional spring of equivalent stiffness $k_{\theta}$ as shown in the figure. The centre of gravity$\brak{\text{ CG }}$ of the wing idealized as an airfoil is marked in the figure. The number of degree $\brak{\text{ s }}$ of freedom for this idealized wing vibration model is $\underline{\hspace{2cm}}.\brak{\text{ Answer in integer }}$
\begin{figure}[H]
\centering
\resizebox{0.3\textwidth}{!}{%
\begin{circuitikz}
\tikzstyle{every node}=[font=\small]
\draw [short] (2.25,10.25) -- (2,10);
\draw [short] (2.5,10.25) -- (2.25,10);
\draw [short] (2.75,10.25) -- (2.5,10);
\draw [short] (3,10.25) -- (2.75,10);
\draw [short] (3.25,10.25) -- (3,10);
\draw [short] (3.5,10.25) -- (3.25,10);
\draw [short] (3.75,10.25) -- (3.5,10);
\draw [short] (4,10.25) -- (3.75,10);
\draw [short] (4,10) -- (4.25,10.25);
\draw [short] (4,9.75) -- (4.5,10);
\draw [short] (4,9.5) -- (4.5,9.75);
\draw [short] (4,9.25) -- (4.5,9.5);
\draw [short] (4,9) -- (4.5,9.25);
\draw [short] (4,8.75) -- (4.5,9);
\draw [short] (4,8.5) -- (4.5,8.75);
\draw [short] (4,8.25) -- (4.5,8.5);
\draw [short] (4,8) -- (4.5,8.25);
\draw [short] (4,7.75) -- (4.5,8);
\draw [short] (4,7.5) -- (4.5,7.75);
\draw [short] (4,7.25) -- (4.5,7.5);
\draw [short] (4,7) -- (4.5,7.25);
\draw [short] (4,6.75) -- (4.5,7);
\draw [short] (4,6.5) -- (4.5,6.75);
\draw (3,10) to[R] (3,6.5);
\draw [short] (1.5,7) .. controls (3.25,7) and (4.25,6.25) .. (5.25,6);
\draw [short] (1.5,7) .. controls (1.75,6.25) and (4.25,5.5) .. (5.25,6);
\draw [short] (4,6.75) -- (3,6.75);
\draw  (3,6.25) circle (0.5cm);
\draw  (3,6.25) circle (0.25cm);
\node at (3,6.5) [circ] {};
\node [font=\small] at (2.5,8.25) {K};
\node [font=\small] at (2.25,5.75) {k${_\theta}$};
\node [font=\large] at (4.25,6) {CG};
\end{circuitikz}
}%

\end{figure}
\bigskip
\item The system of equations\\
$x-2y+\alpha z=0\\2x+y-4z=0\\x-y+z=0$\\
has a non-trivial solution for $\alpha=\underline{\hspace{2cm}}.\brak{\text{ Answer in integer }}$
\bigskip
\item An airplane weighing $40 kN$ is landing on a horizontal runway during which it is retarded by an arresting cable mechanism. The tension in the arresting cable at a given instant, as shown in the figure, is $100 kN.$ Assuming that the thrust from the engine continues to balance airplane drag, the magnitude of horizontal load factor is $\underline{\hspace{2cm}}.\brak{\text{ round off to one decimal place }}$
\begin{figure}[!ht]
\centering
\resizebox{0.5\textwidth}{!}{%
\begin{circuitikz}
\tikzstyle{every node}=[font=\small]
\draw [short] (0,5.25) -- (10.5,5.25);
\draw [short] (1.75,6.5) -- (1.5,5.75);
\draw [short] (6,6.5) -- (6,6);
\draw [short] (1.5,6.5) .. controls (4.5,6.25) and (4.5,6.25) .. (7.5,6.5);
\draw [short] (2.25,7.5) .. controls (4,8) and (5,8.25) .. (7.5,6.75);
\draw [short] (1.5,6.5) .. controls (0.5,6.75) and (0.25,8.25) .. (2.25,7.5);
\draw [short] (7.5,6.75) .. controls (7.5,7) and (8.5,6.5) .. (7.5,6.5);
\draw [short] (6.5,7.25) -- (6.75,8);
\draw [short] (6.75,8) -- (7.25,8);
\draw [short] (7.25,8) -- (7.5,6.75);
\draw  (1.5,5.75) circle (0.5cm);
\draw  (6,5.75) circle (0.5cm);
\draw [short] (7.25,6.5) -- (10.25,5.25);
\draw [short] (9.25,5.25) .. controls (8.75,5.75) and (8.75,6) .. (9,5.75);
\draw [short] (9,5.75) -- (9,5.5);
\draw [short] (9,5.75) -- (8.75,5.75);
\draw [short] (2,7) .. controls (3,7.25) and (3.5,7.5) .. (4.25,7.5);
\draw [short] (2,7) .. controls (3.25,7) and (3.5,7) .. (4.25,7.5);
\node [font=\large] at (8.5,5.5) {10$\circ$};
\node [font=\small] at (9.5,6.25) {Arresting Cable};
\node at (7.25,6.5) [circ] {};
\node at (10.25,5.25) [circ] {};
\end{circuitikz}
}%

\end{figure}
\bigskip
\item The ratio of the speed of sound in H$_{2}\brak{\text{ molecular weight }2\frac{kg}{kmol}}$ to that in N$_{2\brak{\text{ molecular weight }28\frac{kg}{kmol}}}$ at temperature $300\text{ K }$ and pressure $2\text{ bar }$ is $\underline{\hspace{2cm}}.\brak{\text{ round off to two decimal place }}$
\bigskip
\item Airplane A and Airplane B are cruising at altitudes of $2 km \text{ and } 4 km$,respectively. The free stream density and static pressure at altitude $2 km\text{ are } 1.01\frac{kg}{m^{3}}\text{ and } 79.50 kPa$ respectively, and at altitude $4 km \text{ they are }0.82\frac{kg}{m^{3}}\text{ and }61.70kPa,$ respectively. The differential pressure reading from the pitot-static tubes is
$3kpa$ a for both the airplanes. Assuming incompressible flow, the ratio of
cruise speeds of Airplane A to Airplane B is $\underline{\hspace{2cm}}.\brak{\text{ round off to two decimal place }}$
\bigskip
\item A supersonic vehicle powered by a ramjet engine is cruising at a speed of $1000\frac{m}{s}$ The ramjet engine burns hydrogen in a subsonic combustor to produce thrust. The heat of combustion for hydrogen is $120\frac{MJ}{kg}.$ The overall efficiency of the engine $\eta_0,$ defined as the ratio of propulsive power to the total heat release in the combustor, is $40\%$ Taking acceleration due to gravity $g_{0}=10\frac{m}{s^{2}},$ the specific impulse of the engine is
$\underline{\hspace{2cm}}.\text{ seconds }\brak{\text{ round off to two decimal place }}$
\bigskip
\item Given the function $y\brak{x}=\brak{x+3}\brak{x-2},$ for $-4<x<4.$ What is the value of x at which the function has a minimum?
\begin{multicols}{4}
\begin{enumerate}
    \item $\frac{-3}{2}$
    \item $\frac{-1}{2}$
    \item $\frac{1}{2}$
    \item $\frac{3}{2}$
\end{enumerate}
\end{multicols}
\bigskip
\item A supersonic aircraft has an air intake ramp that can be rotated about the leading edge O such that the shock from the leading edge meets the cowl lip as shown in the figure. Select all the correct statement$\brak{\text{ s }}$ as per oblique shock theory when the flight Mach number M is increased.
\begin{figure}[!ht]
\centering
\resizebox{0.5\textwidth}{!}{%
\begin{circuitikz}
\tikzstyle{every node}=[font=\small]
\draw (0.75,10.75) to[short] (8.5,10.75);
\draw (0.75,10.75) to[short] (4.25,10.75);
\draw [short] (0.75,10.75) -- (4.25,9.75);
\draw [short] (4.25,9.75) -- (4.25,8.5);
\draw [short] (4.25,8.5) -- (8.75,8.5);
\draw [short] (0.75,10.75) -- (4.25,8.5);
\draw [short] (1.75,10.75) .. controls (2.25,11) and (3.25,10.75) .. (2.25,9.75);
\draw [short] (2,10) -- (2.25,9.75);
\draw [short] (2.25,9.75) -- (2.75,9.75);
\draw [short] (3,10.75) .. controls (4,10.75) and (4,10.75) .. (3.5,10);
\draw [short] (3.25,10.25) -- (3.5,10);
\draw [short] (3.5,10) -- (3.75,10);
\draw [->, >=Stealth] (-0.25,9) -- (2.25,9);
\node [font=\large] at (0.5,10.75) {O};
\node [font=\large] at (0.75,9.5) {M>1};
\node [font=\small] at (3,9.75) {$\beta_{SHOCK}$};
\node [font=\small] at (4.25,10.25) {$\theta_{RAMP}$};
\node [font=\small] at (4.75,8.25) {$COWL LIP$};
\node [font=\small] at (5,9.25) {$INTAKE$};
\end{circuitikz}
}%

\end{figure}
\begin{enumerate}
    \item It is always possible to find a ramp setting $\theta_{RAMP}$ such that the shock still meets the cowl lip $\brak{\beta_{SHOCK}\text{ remains the same }}.$
    \item if $\theta_{RAMP}$ is held fixed, the shock angle $\beta_{SHOCK}$ will increase
    \item If M exceeds a critical value, it would NOT be possible to find a ramp setting $\theta_{RAMP}$ such that the shock still meets the cowl lip $\brak{\beta_{SHOCK}\text{ remains the same }}.$
    \item Shock angle $\beta_{SHOCK}<\sin^{-1}\brak{\frac{1}{M}}$
\end{enumerate}
\bigskip
\item Two missiles A and B powered by solid rocket motors have identical specific
impulse, liftoff mass of $5600$kg each, and burn durations of $t_{A}=30s\text{ and } t_{B}=70s.$ The propellant mass flow rates $\Dot{m}_{A}\text{ and }\Dot{m}_{B},$ for missiles A and B, respectively, are given by\\
$\Dot{m}_{A}=120\frac{kg}{s},0\leq t\leq30$\\
$\Dot{m}_{B}=70\frac{kg}{s},0\leq t\leq70$\\
Neglecting gravity and aerodynamic forces, the relationship between the final
velocities $V_{A}\text{ and }V_{B}$ of missiles A and B, respectively, is given by
\begin{enumerate}
    \item $V_{A}=4.1V_{B}$
    \item $V_{A}=V_{B}$
    \item $V_{A}=0.5V_{B}$
    \item $V_{A}=0.7V_{B}$
\end{enumerate}
\bigskip
\item A perfect gas stored in a large reservoir exhausts into the atmosphere through a convergent duct. The reservoir pressure is $P_{0}$ and temperature is $T_{0}.$  The jet emerges from the nozzle at choked conditions with average velocity u, Mach number M, pressure p, temperature T, and density $\rho.$ If the reservoir pressure is increased, then
\begin{enumerate}
    \item u, M, p, T, and $\rho$ increase
    \item u,  p, T, and $\rho$ increase, but M remains the same 
    \item u, M, and T remain the same, but p and $\rho$ increase
    \item u, M,  T, and $\rho$ remain the same, but p increase
\end{enumerate}



\end{enumerate}

\end{document}
