\let\negmedspace\undefined
\let\negthickspace\undefined
\documentclass[journal]{IEEEtran}
\usepackage[a5paper, margin=10mm, onecolumn]{geometry}
%\usepackage{lmodern} % Ensure lmodern is loaded for pdflatex
\usepackage{tfrupee} % Include tfrupee package

\setlength{\headheight}{1cm} % Set the height of the header box
\setlength{\headsep}{0mm}     % Set the distance between the header box and the top of the text

\usepackage{gvv-book}
\usepackage{gvv}
\usepackage{cite}
\usepackage{amsmath,amssymb,amsfonts,amsthm}
\usepackage{algorithmic}
\usepackage{graphicx}
\usepackage{textcomp}
\usepackage{xcolor}
\usepackage{txfonts}
\usepackage{listings}
\usepackage{enumitem}
\usepackage{mathtools}
\usepackage{gensymb}
\usepackage{comment}
\usepackage[breaklinks=true]{hyperref}
\usepackage{tkz-euclide} 
\usepackage{listings}

% \usepackage{gvv}                                        
\def\inputGnumericTable{}                                 
\usepackage[latin1]{inputenc}                                
\usepackage{color}                                            
\usepackage{array}                                            
\usepackage{longtable}                                       
\usepackage{calc}                                             
\usepackage{multirow}                                         
\usepackage{hhline}                                           
\usepackage{ifthen}                                           
\usepackage{lscape}
\begin{document}

\bibliographystyle{IEEEtran}
\vspace{3cm}


\renewcommand{\thefigure}{\theenumi}
\renewcommand{\thetable}{\theenumi}
\setlength{\intextsep}{10pt} % Space between text and floats


\numberwithin{equation}{enumi}
\numberwithin{figure}{enumi}
\renewcommand{\thetable}{\theenumi}

\title{JEE Mains PYQ 04/04/2024 Shift-1}
\author{AI24BTECH11032 Shreyansh Sonkar
}
\maketitle
\renewcommand{\thefigure}{\theenumi}
\renewcommand{\thetable}{\theenumi}
\begin{enumerate}[start=16] % This line has been modified to start from number 16\\
\item Let the point, on the line passing through the points P$\brak{1,-2,3}$ and Q$\brak{5,-4,7},$ farther from the origin and at a distance of $9$ units from the point P, be $\brak{\alpha,\beta,\gamma}.$ Then $\alpha^{2}+\beta^{2}+\gamma^{2}$ is equal to :
\begin{multicols}{2}
    \begin{enumerate}
        \item $155$
        \item $150$
        \item $160$
        \item $165$
    \end{enumerate}
\end{multicols}
\bigskip
\item A square is inscribed in the circle $x^{2}+y^{2}-10x-6y+30=0.$ One side of this square is parallel to $y = x + 3. $
If $\brak{x_i,y_i}$ are the vertices of the square , then $\sum \brak{x_i^{2}+y_i^{2}}$ is equal to:
\begin{multicols}{2}
    \begin{enumerate}
        \item $148$
        \item $156$
        \item $160$
        \item $152$
    \end{enumerate}
\end{multicols}
\bigskip
\item If the domain of the function $\sin^{-1}\brak{\frac{3x - 22}{2x - 19}} + \log_e\brak{\frac{3x^2 - 8x + 5}{x^2 - 3x - 10}}$ is $(\alpha,\beta]$ then $3\alpha+10\beta$ is equal to:
\begin{multicols}{2}
    \begin{enumerate}
        \item $97$
        \item $100$
        \item $95$
        \item $98$
    \end{enumerate}
\end{multicols}
\bigskip
\item Let$f\brak{x}=x^{5}+2e^{\frac{x}{4}}$ for all $x \in R.$ Consider function $g\brak{x}$ such that $\brak{gof}\brak{x}=x$ for all $x \in R.$ Then the value of $8g^{'}\brak{2}$ is :
\begin{multicols}{2}
    \begin{enumerate}
        \item $16$
        \item $4$
        \item $8$
        \item $2$
    \end{enumerate}
\end{multicols}
\bigskip
\item Let $\alpha \in \brak{0,\infty}$ and $A = \begin{bmatrix} 
1 & 2 & \alpha \\ 
1 & 0 & 1 \\ 
0 & 1 & 2 
\end{bmatrix}.$ If $\text{det\brak{adj\brak{2A-A^{T}}.adj\brak{A-2A^{T}}}}=2^{8},$ then $\brak{det\brak{A}}^{2}$ is equal to:
\begin{multicols}{2}
    \begin{enumerate}
        \item $1$
        \item $49$
        \item $16$
        \item $36$
    \end{enumerate}
\end{multicols}
\bigskip
\item If $\lim_{x \to 1} \frac{\brak{5x+1}^{\frac{1}{3}}-\brak{x+5}^{\frac{1}{3}}}{\brak{2x+3}^{\frac{1}{2}}-\brak{x+4}^{\frac{1}{2}}}=\frac{m\sqrt{5}}{n\brak{2n}^\frac{2}{3}},$ where $gcd\brak{m,n}=1,$ then $8m+12n$ is equal to.
\bigskip
\item In a survey of $220$ students of a higher secondary
school, it was found that at least $125$ and at most
$130$ students studied Mathematics; at least $85$ and
at most $95$ studied Physics; at least $75$ and at most
$90$ studied Chemistry; $30$ studied both Physics and
Chemistry; $50$ studied both Chemistry and
Mathematics; $40$ studied both Mathematics and
Physics and $10$ studied none of these subjects. Let
m and n respectively be the least and the most
number of students who studied all the three
subjects. Then $m + n$ is equal to
\bigskip
\item Let the solution $y=y\brak{x}$ of the differential equaion $\frac{dy}{dx}-y=1+4sinx$ satisfy $y\brak{\pi}=1.$ Then $y\brak{\frac{\pi}{2}+10}$ is equal to.
\bigskip
\item If the shortest distance between the lines $\frac{x+2}{2}=\frac{y+3}{3}=\frac{z-5}{4}$ and $\frac{x-3}{1}=\frac{y-2}{-3}=\frac{z+4}{2}$ is $\frac{38}{3\sqrt{5}}k$ and $\int_{0}^{k}\sbrak{x^{2}}dx=\alpha-\sqrt{\alpha
},$ where $\sbrak{x}$ denotes the greatest integer function, then $6\alpha^{3}$ is equal to.
\bigskip
\item Let A be a square matrix of order 2 such that $\abs{A}=2$
and the sum of its diagonal elements is $-3.$ If the point$\brak{x,y}$ satisfying $A^{2}+xA+yI=0$ lie
on a hyperbola, whose transverse axis is parallel to
the x-axis, eccentricity is e and the length of the
latus rectum is l,then $e^{4}+l^{4}$ is equal to.
\bigskip
\item let $a=1+\frac{^2C_2}{3!}+\frac{^3C_2}{4!}+\frac{^4C_2}{5!}+\cdots,$
$b=1+\frac{^1C_0+^1C_1}{1!}+\frac{^2C_0+^2C_1+^2C_2}{2!}+\frac{^3C_0+^3C_1+^3C_2+^3C_3}{3!}+\cdots$ Then $\frac{2b}{a^{2}}$ is equal to .
\bigskip
\item Let A be a $3\times3$ matrix of non-negative real
elements such that $A \cdot \begin{bmatrix} 1 \\ 1 \\ 1 \end{bmatrix} = 3 \cdot \begin{bmatrix} 1 \\ 1 \end{bmatrix}
$ Then the maximum value of $det\brak{A}$ is.
\bigskip
\item Let the length of the focal chord PQ of the
parabola $y^{2} = 12x$ be 15 units. If the distance of PQ
from the origin is p, then $10p^{2}$ is equal to.
\bigskip
\item Let ABC be a triangle of area $15\sqrt{2}$ and the vectors $\overrightarrow{AB}=\hat{i}+2\hat{j}-7\hat{k},$ $\overrightarrow{BC}=a\hat{i}+b\hat{j}+c\hat{k}$ and $\overrightarrow{AC}=6\hat{i}+d\hat{j}-2\hat{k},$ $d>0.$ Then The square of the length of the largest side of the triangle ABC is .
\bigskip
\item If $\int_{0}^{\frac{\pi}{4}}\frac{sin^{2}x}{1+sinxcosx}dx=\frac{1}{a}log_e\brak{\frac{a}{3}}+\frac{\pi}{b\sqrt{3}},$ where $a,b\in N ,$ then $a+b$ is equal to. 
\end{enumerate}
\end{document}
