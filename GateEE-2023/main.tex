\let\negmedspace\undefined
\let\negthickspace\undefined
\documentclass[journal]{IEEEtran}
\usepackage[a5paper, margin=10mm, onecolumn]{geometry}
%\usepackage{lmodern} % Ensure lmodern is loaded for pdflatex
\usepackage{tfrupee} % Include tfrupee package

\setlength{\headheight}{1cm} % Set the height of the header box
\setlength{\headsep}{0mm}     % Set the distance between the header box and the top of the text

\usepackage{gvv-book}
\usepackage{gvv}
\usepackage{cite}
\usepackage{amsmath,amssymb,amsfonts,amsthm}
\usepackage{algorithmic}
\usepackage{graphicx}
\usepackage{textcomp}
\usepackage{xcolor}
\usepackage{txfonts}
\usepackage{listings}
\usepackage{enumitem}
\usepackage{mathtools}
\usepackage{gensymb}
\usepackage{comment}
\usepackage[breaklinks=true]{hyperref}
\usepackage{tkz-euclide} 
\usepackage{listings}
% \usepackage{gvv}                                        
\def\inputGnumericTable{}                                 
\usepackage[latin1]{inputenc}                                
\usepackage{color}                                            
\usepackage{array}                                            
\usepackage{longtable}                                       
\usepackage{calc}                                             
\usepackage{multirow}                                         
\usepackage{hhline}                                           
\usepackage{ifthen}                                           
\usepackage{bookmark}
\usepackage{lscape}

\begin{document}

\bibliographystyle{IEEEtran}
\vspace{3cm}

\title{Gate EE-2023}
\author{AI24BTECH11032 Shreyansh Sonkar}
% \maketitle
% \newpage
% \bigskip
{\let\newpage\relax\maketitle}

\renewcommand{\thefigure}{\theenumi}
\renewcommand{\thetable}{\theenumi}
\setlength{\intextsep}{10pt} % Space between text and floats


\numberwithin{equation}{enumi}
\numberwithin{figure}{enumi}
\renewcommand{\thetable}{\theenumi}

\begin{enumerate} [start=40]
\item An $8$ bit ADC converts analog voltage in the range of 0 to $+5$ V to the corresponding
digital code as per the conversion characteristics shown in figure.
For $V_{in} = 1.9922 V$, which of the following digital output, given in hex, is true ?
\begin{figure}[!ht]
\centering
\resizebox{0.3\textwidth}{!}{%
\begin{circuitikz}
\tikzstyle{every node}=[font=\normalsize]
\draw (3.5,10.75) to[short] (3.5,5.25);
\draw (3.5,5.25) to[short] (10.5,5.25);
\draw [ line width=0.9pt](3.5,5.25) to[short] (4,5.25);
\draw [ line width=0.9pt](4,5.25) to[short] (4,6);
\draw [ line width=0.9pt](4,6) to[short] (5,6);
\draw [ line width=0.9pt](5,6) to[short] (5,6.75);
\draw [ line width=0.9pt](5,6.75) to[short] (5.75,6.75);
\draw [ line width=0.9pt](5.75,6.75) to[short] (5.75,7.5);
\draw [ line width=0.9pt](5.75,7.5) to[short] (6.5,7.5);
\draw [ line width=0.9pt](6.5,7.5) to[short] (6.5,8.25);
\draw [ line width=0.9pt](6.5,8.25) to[short] (7.25,8.25);
\draw [ line width=0.9pt](7.25,8.25) to[short] (7.25,9);
\draw [ line width=0.9pt](7.25,9) to[short] (8,9);
\draw [line width=0.9pt, ->, >=Stealth, dashed] (3.5,4.25) -- (3.5,5.25);
\draw [line width=0.9pt, ->, >=Stealth, dashed] (4,4.75) -- (4,5.25);
\draw [line width=0.9pt, ->, >=Stealth, dashed] (4.5,4.25) -- (4.5,6);
\draw [line width=0.9pt, ->, >=Stealth, dashed] (5.5,4.25) -- (5.5,6.75);
\draw [line width=0.9pt, ->, >=Stealth, dashed] (6.25,4.25) -- (6.25,7.5);
\draw [line width=0.9pt, ->, >=Stealth, dashed] (7,4.25) -- (7,8.25);
\draw [dashed] (3.5,6) -- (4,6);
\draw [dashed] (5,6.75) -- (3.5,6.75);
\draw [dashed] (5.75,7.5) -- (3.5,7.5);
\draw [dashed] (7.75,5) -- (9.5,5);
\node [font=\small] at (2.75,9.25) {Digital};
\node [font=\small] at (2.75,9) {output in};
\node [font=\small] at (2.5,8.75) {Hex};
\draw [dashed] (3,8.75) -- (3,8.25);
\node [font=\small] at (3.25,7.5) {0.3H};
\node [font=\small] at (3.25,6.75) {02H};
\node [font=\small] at (3.25,6) {01H};
\node [font=\small] at (3.25,5.25) {00H};
\node [font=\small] at (10.25,5) {$V_{in} in mV$};
\node [font=\small] at (9.5,4.5) {$Analog Input $};
\node [font=\large] at (3.5,4.75) {0};
\node [font=\normalsize] at (4,4.5) {9.8};
\node [font=\normalsize] at (4.5,5) {19.6};
\node [font=\normalsize] at (5.5,5) {39.2};
\node [font=\normalsize] at (6.25,5) {58.8};
\end{circuitikz}
}%
\end{figure}
\begin{enumerate}
    \item $64$H
    \item $65$H
    \item $66$H
    \item $67$H
\end{enumerate}
\bigskip
\item The three-bus power system shown in the figure has one alternator connected to
bus $2$ which supplies $200$ MW and $40$ MVAr power. Bus $3$ is infinite bus having a voltage of magnitude $\abs{V_{3}}=1.0$p.u and angle of $-15^{\circ}.$ A variable current source, $\abs{I}\angle\phi$ is connected at bus $1$ and controlled such that the magnitude of the bus $1$
voltage is maintained at$ 1.05$ p.u. and the phase angle of the source current $\phi=\theta_{1}\pm\frac{\pi}{2},\text{ where }\theta_{1}$ is the phase angle of the bus $1$ voltage. The three buses can be categorized for load flow analysis as

\begin{figure}[!ht]
\centering
\resizebox{0.5\textwidth}{!}{%
\begin{circuitikz}
\tikzstyle{every node}=[font=\small]

% Main vertical lines
\draw [line width=0.9pt] (2.5,11) -- (2.5,5.75);
\draw [line width=0.9pt] (12.75,11) -- (12.75,5.75);

% Inductors
\draw [line width=0.9pt] (2.5,10.5) to[L=$j0.4$] (12.75,10.5);
\draw [line width=0.9pt] (2.5,6) to[L=$j0.3$] (6.75,6);
\draw [line width=0.9pt] (6.75,6) to[L=$j0.3$] (12.75,6);

% Voltage Source and Arrows
\draw [line width=0.5pt] (2.5,8) to[sinusoidal voltage source, sources/symbol/rotate=auto] (0,8);
\draw [line width=0.5pt, ->, >=Stealth] (1.25,8.75) -- (2.25,8.75);

% Current source and connections
\draw [line width=0.5pt] (7.5,6.5) -- (7.5,5.25);
\draw [line width=0.5pt] (7.5,5.5) -- (8,5.5) -- (8,4.75);
\draw [line width=0.5pt] (8,4.5) circle (0.25cm); % Controlled current source
\draw [line width=0.5pt, ->, >=Stealth] (8,4.5) -- (8,4.75);
\draw [line width=0.5pt] (7.25,4) -- (7.75,4.5);
\draw [line width=0.5pt] (8,4.25) -- (8,4) node[ground]{};
\draw [line width=0.5pt, ->, >=Stealth] (8,4.75) -- (8.5,5.25);

% Infinite bus and connections
\draw [line width=0.5pt] (12.75,8) -- (14,8);
\draw [line width=0.5pt] (14.5,8) circle (0.5cm); % Infinite bus symbol

% Nodes
\draw [line width=0.5pt] (2.5,11.75) circle (0.25cm); % Node 2
\draw [line width=0.5pt] (12.5,11.75) circle (0.25cm); % Node 3
\draw [line width=0.5pt] (7.5,7) circle (0.25cm); % Node 1

% Labels
\node [font=\small] at (0.75,10.25) {$P_2=200\, \text{MW}$};
\node [font=\small] at (0.75,9.75) {$Q_2=40\, \text{MVAr}$};
\node [font=\small] at (0.5,7) {Alternator};
\node [font=\small] at (1,6) {$|V_{2}|\angle\theta$};
\node [font=\small] at (5.25,4.5) {Controlled};
\node [font=\small] at (5,4.25) {Current};
\node [font=\small] at (5,4) {Source};
\node [font=\small] at (7.25,7.75) {$V_{1}=1.05\angle\theta_{1}$};
\node [font=\small] at (7.5,7) {1};
\node [font=\small] at (2.5,11.75) {2};
\node [font=\small] at (12.5,11.75) {3};
\node [font=\small] at (13.5,6.75) {Infinite Bus};
\node [font=\small] at (14.25,9.25) {$|V_{3}|\angle\theta_{3}$ =};
\node [font=\small] at (14.25,9) {$1.0\angle-15^{\circ}$};
\node [font=\small] at (9.25,4.75) {$|\mathbf{I}|\angle\phi$};
\node [font=\small] at (9.75,4.25) {where $\phi=\theta_{1}\pm \frac{\pi}{2}$};

\end{circuitikz}
}
\end{figure}

\begin{enumerate}
    \item Bus 1  Slack bus\\ Bus 2  $P-\abs{V}\text{ bus }$\\Bus 3  $P-Q \text{ bus }$
    \item Bus 1  $P-\abs{V}\text{ bus }$\\Bus 2  $P-\abs{V}\text{ bus }$\\Bus 3  Slack bus
    \item Bus 3  $P-Q \text{ bus }$\\Bus 2  $P-Q \text{ bus }$\\Bus 3  Slack bus
    \item Bus 1  $P-\abs{V}\text{ bus }$\\Bus 2  $P-Q \text{ bus }$\\Bus 3  Slack bus
\end{enumerate}
\bigskip
\item Consider the following equation in a 2-D real-space.\\
$\abs{x_{1}}^{p}+\abs{x_{2}}^{p}=1\text{ for } p>0$\\
Which of the following statement$\brak{\text{ s }}$ is/are true.
\begin{enumerate}
    \item When $p = 2$, the area enclosed by the curve is $\pi.$
    \item When p tends to$\infty$ the area enclosed by the curve tends to $4$.
    \item When p tends to $0,$ the area enclosed by the curve is$ 1.$
    \item When $p = 2$, the area enclosed by the curve is $2.$
\end{enumerate}
\bigskip
\item In the figure , the electric field E and the magnetic field B point to x and z directions, respectively  and have constant magnitude . A positive charges 'q' is released  from  rest at the origin. Which of the following statement$\brak{\text{ s }}$ is/are true
\begin{figure}[!ht]
\centering
\resizebox{0.5\textwidth}{!}{%
\begin{circuitikz}
\tikzstyle{every node}=[font=\small]
\draw [line width=0.5pt, ->, >=Stealth] (3,9) -- (1.25,5.25);
\draw [line width=0.5pt, ->, >=Stealth] (3,9) -- (10.75,9);
\draw [line width=0.5pt, ->, >=Stealth] (3,9) -- (3,12.75);
\draw [line width=1.2pt, ->, >=Stealth] (3.5,10.5) -- (3.5,11.75);
\draw [line width=1.2pt, ->, >=Stealth] (3,8) -- (2.5,7);
\node [font=\small] at (2.5,11.75) {z};
\node [font=\small] at (4,11) {E};
\node [font=\small] at (1.25,6.5) {x};
\node [font=\small] at (3,7.5) {B};
\node [font=\small] at (3.25,8.75) {q};
\node [font=\small] at (10.25,8.75) {y};
\end{circuitikz}
}%

\end{figure}
\begin{enumerate}
    \item The charge will move in the direction of z with constant velocity.
    \item The charge will always move on the y-z plane only.
    \item The trajectory of the charge will be a circle.
    \item The charge will progress in the direction of y.
\end{enumerate}
\bigskip
\item All the elements in the circuit shown in the following figure are ideal. Which of the
following statements is/are true?
\begin{figure}[H]
\centering
\resizebox{0.5\textwidth}{!}{%
\begin{circuitikz}
\tikzstyle{every node}=[font=\large]
\draw [ line width=1.2pt](1.5,11) to[american current source] (4.5,11);
\draw (1.5,11) to[battery2] (1.5,7.25);
\draw [ line width=1.2pt](4.5,11) to[D] (9.5,11);
\draw (9.5,11) to[battery2] (9.5,7.25);
\draw [ line width=1.2pt](9.5,11) to[american current source] (13,11);
\draw [ line width=1.2pt](13,11) to[D] (16.75,11);
\draw (16.75,11) to[battery2] (16.75,7.25);
\draw [ line width=1.2pt](1.5,7.25) to[short] (16.75,7.25);
\draw [ line width=1.2pt](13.25,11) to[short, -o] (13.25,9.5) ;
\draw [ line width=1.2pt](13.25,7.25) to[short, -o] (13.25,8.75) ;
\draw [ line width=0.6pt](13.25,9.5) to[short] (13.75,9);
\draw [ line width=0.6pt](5.25,11) to[short] (5.25,13);
\draw [ line width=0.6pt](5.25,13) to[D] (13.25,13);
\draw [ line width=0.6pt](13.25,13) to[short] (13.25,11);
\node [font=\small] at (16.5,9.5) {+};
\node [font=\normalsize] at (9.25,9.5) {+};
\node [font=\normalsize] at (1.25,9.5) {+};
\node [font=\large] at (10.25,9) {20 V};
\node [font=\large] at (7,10.5) {D${_2}$};
\node [font=\large] at (3,11.75) {4 A, DC};
\node [font=\large] at (9.25,13.75) {D${_1}$};
\node [font=\large] at (11.25,11.75) {2 A,DC};
\node [font=\large] at (15,11.5) {D${_3}$};
\node [font=\large] at (17.25,9.5) {40 V};
\node [font=\large] at (13,9) {S};
\node [font=\large] at (0.75,9) {10 V};
\end{circuitikz}
}%

\end{figure}
\begin{enumerate}
    \item When switch S is ON, both $D_{1} \text{ and } D_{2} $conducts and $D_{3}$ is reverse biased
    \item When switch S is ON, $D_{1}$ conducts and both $D_{1} \text{ and } D_{2} $ are reverse biased
    \item When switch S is OFF,$D_{1}$ is reverse biased and both $D_{1} \text{ and } D_{2} $  conduct
    \item When switch S is OFF, $D_{1}$ conducts, $D_{2}$ is reverse biased and $D_{3}$ conducts
\end{enumerate}
\bigskip
\item The expected number of trails for first occurrence  of a $" \text{ head }"$ in a biased coin is known to be $4.$ The probability of first occurrence of a $" \text{ head }"$ in the second trial is $\underline{\hspace{2cm}}.\brak{\text{ Round off to three decimal places}}$
\bigskip
\item Consider the state-space description of an LTI system with matrices
\begin{align*}
    A = \begin{bmatrix}
0 & 1 \\
-1 & -2 \\
\end{bmatrix}, \quad
B = \begin{bmatrix}
0 \\
1 \\
\end{bmatrix}, \quad
C = \begin{bmatrix}
3 & -2 \\
\end{bmatrix}, \quad
D = 1
\end{align*}
for the input $\sin\brak{\omega t},\omega>0$ the value of$\omega$ for which the steady-state output of the system will be zero, is $\underline{\hspace{2cm}}.\brak{\text{ Round off to the nearest integer}}$
\bigskip
\item A three-phase synchronous motor with synchronous impedance of $0.1+j0.3$ per unit per phase has a static stability limit of $2.5$ per unit. The corresponding excitation voltage in per unit is $\underline{\hspace{2cm}}.\brak{\text{ Round off to two decimal place}}$
\bigskip
\item A three phase $415 \text{ V }, 50 \text{ Hz }, 6-\text{ pole }, 960 \text{ RPM }, 4 \text{ HP }$ squirrel cage induction motor drives a constant torque load at rated speed operating from rated supply and delivering rated output. If the supply voltage and frequency are reduced by $20\%,$ the resultant speed of the motor in RPM
$\brak{\text{ neglecting the stator leakage impedance and rotational losses}}$ is
$\underline{\hspace{2cm}}.\brak{\text{ Round off to the nearest integer}}$
\bigskip
\item The period of the discrete-time signal x$\sbrak{n}$ described by the equation below is N $=\underline{\hspace{2cm}}.\brak{\text{ Round off to the nearest integer}}$
\begin{align*}
    x\sbrak{n}=1+3\sin\brak{\frac{15\pi}{8}n+\frac{3\pi}{4}}-5\sin\brak{\frac{\pi}{3}n-\frac{\pi}{4}}
\end{align*}
\bigskip
\item The discrete-time Fourier transform of a signal x$\sbrak{n}$ is X$\brak{\ohm}=\brak{1+\cos\ohm}e^{-j\ohm}$. Consider that x$x_{p}\sbrak{n}$ is a periodic signal of period N$=5$ such that 
\begin{align*}
    x_{n}\sbrak{n}=x\sbrak{n},\text{ for } n=0,1,2\\
    =0,\text{ for } n=3,4
\end{align*}
Note that $x_{p}\sbrak{n}=\sum_{k=0}^{N-1}\alpha_{k}e^{j^{\frac{2\pi}{N}kn}}.$ The magnitude of the Fourier series coefficient $\alpha_{3}$ is $=\underline{\hspace{2cm}}.\brak{\text{ Round off to three decimal places}}$
\bigskip
\item For the circuit shown, if $i=\sin1000t,$ the instantaneous value of the Thevenin's equivalent voltage $\brak{\text{ in Volts }}$ across the terminals $a-b$ at time $t=5$ is $=\underline{\hspace{2cm}}.\brak{\text{ Round off to two decimal places}}$
\begin{figure}[!ht]
\centering
\resizebox{0.5\textwidth}{!}{%
\begin{circuitikz}
\tikzstyle{every node}=[font=\large]
\draw [ line width=0.6pt](2.5,11.75) to[american controlled voltage source] (12.75,11.75);
\draw [ line width=0.6pt](12.75,11.75) to[short] (12.75,10);
\draw [ line width=0.6pt ] (12.75,9.5) circle (0.5cm);
\draw [ line width=0.6pt](12.75,9) to[short] (12.75,7.25);
\draw [ line width=0.6pt](12.75,7.25) to[short] (2.75,7.25);
\draw [line width=0.6pt, ->, >=Stealth] (12.75,9) -- (12.75,9.75);
\draw [ line width=0.6pt](10,11.75) to[R] (10,9.25);
\draw [line width=0.6pt](10,9.25) to[L ] (10,7.25);
\draw [ line width=0.6pt](5.75,11.75) to[R] (5.75,9.25);
\draw [line width=0.6pt](5.75,9.25) to[curved capacitor] (5.75,7.25);
\node [font=\large] at (2,11.75) {a};
\node [font=\large] at (2.25,7.25) {b};
\node [font=\large] at (5,10.5) {10$\ohm$};
\node [font=\large] at (4.75,8.25) {$-j 10\ohm$};
\node [font=\large] at (8.75,8.25) {j 10$\ohm$};
\node [font=\large] at (9,10.25) {10$\ohm$};
\node [font=\large] at (14.5,9.5) {i=$\sin1000t$};
\node [font=\large] at (6.75,12.25) {4$i{_x}$};
\node [font=\large] at (6.75,10.25) {i${_x}$};
\draw [line width=0.6pt, ->, >=Stealth] (6.25,11) -- (6.25,9.5);
\end{circuitikz}
}%
\end{figure}
\bigskip
\item The admittance parameters of the passive resistive two-port network shown in the figure are
\begin{align*}
    y_{11}=5S,y_{22}=1,y_{12}=y_{21}=-2.5S
\end{align*} 
The power delivered to the load resistor R$_{L}$ in Watt is $=\underline{\hspace{2cm}}.\brak{\text{ Round off to two decimal places}}$
\begin{figure}[!ht]
\centering
\resizebox{0.5\textwidth}{!}{%
\begin{circuitikz}
\tikzstyle{every node}=[font=\large]
\draw [ line width=0.6pt ] (5.75,11.25) rectangle (8.75,9.25);
\draw [ line width=0.6pt](2.25,11) to[short] (5.75,11);
\draw [ line width=0.6pt](8.75,11) to[short] (11.5,11);
\draw (2.25,11) to[battery ] (2.25,7.25);
\draw [ line width=1.2pt](11.5,11) to[R] (11.5,7.25);
\draw [ line width=1.2pt](2.25,7.25) to[short] (11.5,7.25);
\draw [ line width=1.2pt](4,11) to[short] (4,12.75);
\draw [ line width=1.2pt](10.25,11) to[short] (10.25,12.75);
\draw [ line width=1.2pt](4,12.75) to[R] (10.25,12.75);
\draw [ line width=1.2pt](7.25,9.25) to[short] (7.25,7.25);
\node at (7.25,7.25) [circ] {};
\node at (4.75,11) [circ] {};
\node at (9.5,11) [circ] {};
\node [font=\large] at (12.25,9.5) {R${_L}=$};
\node [font=\large] at (12.5,9) {6$\ohm$};
\node [font=\large] at (1.25,9.25) {20 V};
\node [font=\large] at (7,10.75) {Resistive};
\node [font=\large] at (7,10.25) {Two-port};
\node [font=\large] at (7,9.75) {Network};
\node [font=\large] at (4.75,10.75) {1};
\node [font=\large] at (9.5,10.75) {2};
\node [font=\large] at (7,12) {3$\ohm$};
\node [font=\large] at (7.25,6.75) {$1^{\prime}
,2^{\prime}$};
\end{circuitikz}
}%

\end{figure}




\end{enumerate}

\end{document}
